\chapter{Introduction}
\label{ch:intro}
Cyber-infrastructural systems are getting more integrated with our day-to-day life and have more impact on society, thus, there is a need for approaches to make sure that these systems are operating in a way that is more aligned with norms of society (or policies, rules, regulations). This means norms and policies should be considered as part of infrastructure development and maintenance cycles by the designers. On the other hand, infrastructures affect society and by extension, they are also affecting the norms and how that society is governed, so they should be already considered when policymakers and governance bodies are analyzing existing norms or implementing new policies and services. Taking modelling and model execution as a principal approach to analysing a system, intuitively, requires the experts to be able to create models of these concepts, which include executable models of the infrastructure, relevant norms, and, the social setting and its actors. The expressivity and flexibility required to encompass these aspects are not trivial. It is no longer just a model of infrastructure, social setting, and set of norms, but, it is an infrastructure that is utilized by social actors and governed by norms, a social setting that utilizes and monitors the infrastructure, and decides upon, modifies, analyzes and enforces the norms, and, norms that regulate the infrastructure and the social setting and have impacts on both.


The purpose of this thesis is to closely study the interactions between these concepts, and search for fundamental gaps in current methodologies utilized by system designers, policymakers, and governance bodies to find approaches for creating more expressive and tangible models that are also flexible enough for specifying different scenarios and use-cases in different domains. To propose realistic methods, aside from the theoretical analysis, proof-of-concept tools have also been developed or adopted at each step and are presented.


\paragraph{Social Norms and Behavioral Patterns}
A central theme in this thesis is norms and their relation with individual behavioral patterns. The definition of norms for specific use cases and domains is specified in corresponding chapters, but, it is valuable to set the tone on the complexities of this concept. The daily social life of a human is filled with objectively complex interactions, but somehow we can navigate these dynamic and nondeterministic interactions with a sufficiently low allocation of processing resources. A concept that aids us in doing so is norms, by defining what should and should not be the case and what is expected to happen or not to happen in certain contexts and under certain conditions, norms drive the actions in our fairly complex interactions. 


The fascinating thing about norms is that in any given situation, all parties involved are (somewhat) internally aware of the process and they can act accordingly. We can even deduce this (most probably) correct process in face of highly dynamic and nondeterministic situations with a multitude of variables. It seems like somehow we have converged very closely to a set of abstract, parameterized, context-based, and flexible scenario templates for interactions defined by norms that we use regularly as a coordination mechanism in society. These scenario templates have placeholders for different roles each with its own script, and most of the time we can (almost) precisely interpret a situation, select a relevant and applicable template and instantly fill the parameters and role placeholders with qualified values and infer some sort of decision tree that tells us how interactions can and should move forward. We even monitor and observe events and (own or others') actions and qualify them as traversal steps in these scenarios. Furthermore, concerning our part in a scenario, we seem to have internalized behavioral specifications or partial plans that allow us to enact and embody the role that we have filled with ourselves.


Interpreting the context, selecting a relevant and applicable scenario template, filling in the variables, and qualifying observation are the main components of utilizing scenario templates defined by norms. 

Take a simplistic example that is later on used in this thesis namely a sale transaction; regardless of the specific contexts, the norms for a typical sale transaction specify (almost globally accepted) predefined roles (buyer/seller), variables (item/price) and actions (offer/accept/pay/deliver) related to it. For an example of interpreting the context and selecting an applicable scenario take in-person shopping versus online shopping, while there are obvious differences in the variables and actions involved, we can think of them as the same thing that is a sale transaction with the same abstract template. For an example of filling in the variables, we can easily qualify actors and put them in proper roles the role of the seller in a sale transaction can be filled with ourselves, another person, a company, or more interestingly even an automated software in the form of a webshop. The same level of abstraction also applies to qualifying observations and expectations, receiving cash from an in-person buyer can be qualified as the same action as a notification of a transfer from a banking infrastructure, and the expectation of receiving an item in the next few moments from an in-person seller is the same as the expectations of receiving an oversea delivery in the next few weeks.

Scenario templates defined by norms are also highly dynamic and flexible. During an interaction, we are constantly monitoring the scenario for deviations from what we expect to occur. Some deviations can be negligible: while somewhat unexpected, they do not change the context of the interaction completely, like not receiving a delivery when it was supposed to arrive, or maybe there was a delay. However, some deviations completely change the context, like never receiving an item you paid for and not being able to contact the seller. What is interesting about deviations is that we even have modular and predefined contingency plans for when a concrete scenario is not played according to the expectations of its template. These plans can vary from attempting to fix the concrete scenario, like waiting a bit longer than expected for delivery, to revising some beliefs that resulted in the unexpected situation like removing a particular actor from the qualifications that make them suitable for the seller role to even rewriting the qualification or behavioral rules altogether like never buying anything from any unrecognized webshops.

\paragraph{Regulations, Compliance, and Governance}
Regulations are a form of norms that are (partially) concretized, written as a normative text, and enforced by some authoritative body that governs a system.  Cyber-infrastructural systems that affect members of society ideally should be fully compliant with the norms and regulations of that society. There are multiple ways to ensure compliance in a system, depending on its size, complexity, norms that govern it, and the governing bodies that are concerned with monitoring and regulating it. Some policies can be operationalized in terms of access control. For example, some infrastructures with pre-defined and static internal policies can be developed with the policies already hard-coded in them, intuitively, this is the most efficient approach whilst being the least scalable and maintainable. Some systems use run-time policy-making protocols like LDAP to make sure the infrastructure is running in a compliant manner by e.g., taking permission from policy enforcement points that take advice from policy decision points which in turn retrieve policies from policy information points. Then, by modifying the policies in information points, the administrators can control enforced policies without changing the infrastructure itself. 

These approaches can be referred to as \textit{compliant by design} and are arguably suitable for many situations. That is, until the point that norms in question become more complicated than simple non-conflicting, non-contradicting, easy-to-interpret regulative norms defined in terms of deontic notions of permissions, obligations, and prohibitions implemented in the form of ex-ante access rules. As is observed in the literature --and illustrated with multiple examples in this thesis -- norms do indeed become more complex much sooner than anticipated by most studies, and in fact, are vastly simplified in current compliance-checking approaches. 


Norms are more than a set of formal rules extracted from a legislative text: they emerge from multiple sources with different degrees of priority, and they require interpretation to be encoded and qualification to be applied within a social context. Furthermore, they continuously adapt, in both expression and application~\cite{Boella2014APractice}. Also, in any given context multiple normative sources may be concurrently relevant, and/or multiple interpretations of the same normative sources may be available (e.g. retrieved from previous cases), and these may be possibly conflicting. Finally, not all regulations are ex-ante norms about controlling actions prior to the performance of the action, instead, they are ex-post rules about what ought to be if certain actions have been performed already.% This makes conflict resolution and interpretation and concepts like consequences of actions detrimental parts of compliance-checking.


Furthermore, norms are traditionally distinguished between \textit{regulative} and \textit{constitutive} norms~\cite{Searle1969,Boella2004RegulativeSystems,Sileno2015}. Regulative norms regulate behaviors that exist independent of the norms and are generally prescribed in terms of \textit{permissions}, \textit{obligations} and, \textit{prohibitions} (e.g. traffic regulations). Constitutive norms determine that some entity (e.g. an object, a situation, a certain agent, a certain behaviour) ``counts as'' something else, creating a new institutional entity that does not exist independently of these norms. (for example, the concept of marriage or money as a legal means of payment). The concept of institutional power is particularly relevant in the context of constitutive norms, as it is used to ascribe institutional meaning to performances (e.g. raising a hand counts as a bid during an auction).

% A conceptual framework that instead contains both deontic and potestative dimensions is the one proposed by Hohfeld \cite{hohfeld1917fundamental}, whereas deontic logic, although much more popular, by definition focuses on regulative norms. %This work tries to take a step in the direction of taking these concepts into account in the process of designing compliant systems, or, modifying existing systems to be compliant.

These complexities in norms alongside the fact that cyber-infrastructural systems also include human actors and institutions that are certainly not fully compliant (and arguably, not even benevolent) means that such systems can not be fully compliant by design. Instead, such systems require \textit{governance}~\cite{vanengers2010egoverment}, which includes creation of new policies and regulations or modifying existing ones, enforcing these policies, providing supportive services, monitoring, analysing and imposing punishments and sanctions where needed. Intuitively, if components of the system can be feasibly compliant by design, that is a desirable attribute. However, when the governed system has rapidly changing technological components in it, there will be a need for approaches in adaptation of the governance that can keep up with these changes, and this thesis argues that modelling is a feasible approach. 


\paragraph{Models of Agency}
Agency is another concept that is central to this work. Software systems are generally considered to be void of real agency (specifically in legal terms) and are controlled through specific control structures, even if those structures consist of probabilistic flows that can change as the system interacts with its environment like in case of reinforcement learning they still have a control structure. This assumption becomes fuzzy when it comes to computational models of social actors as these models need to take agency into account, like executable models that should represent humans or organizations, as without access to fully developed artificial general intelligence, it is hard to claim that a software agent has agency. 

Then, the question is how do we model such actors? There are two main approaches used in the field to model concepts of agency such as objectives and preferences, namely these approaches are (1) modelling agents in terms of variables and mathematical functions, and, (2) modelling agents with means of agent-oriented programs often in the form of logic-based rational behavioral specifications, that often are deployed in a Multi-Agent System (MAS).


The first approach has many advantages, firstly, mathematical models are often more efficient in execution resulting in them being more scalable. Also, defining agents in terms of a few variables makes them easier for experimentation particularly in simulations, and, it also makes them much easier to manipulate at run-time, to simulate adaptation and learning processes. 

In general, when the purpose of a research is to study the outcome of social phenomena in certain settings without regards to the model of the input agent, mathematical models are suitable. An example of these approaches is the simulations of disaster response and relief, where purpose of the study is to find the optimal settings or infrastructures to minimize damage. In these cases the agents ---social or infrastructural--- are generally considered to have mathematical specifications that encompass a statistically realistic representation of how a civilian, a disaster response team, or an infrastructure will or should act in case of an incident or natural disaster as their utility function~\cite{arinta2019disaster,khouj2011disaster,chamola2021disaster}, keeping higher level concepts like purpose or intentions implicit in the models.


Where mathematical models fall short and programmable models are advantageous is studies where expressivity and transparency of the models is the main point. In these cases, even if the execution is less efficient and adaptability is harder to achieve, it is still worth to have self-explanatory and readable models of agents that have explicit notions of agency such as intentions and act in a configurable social setting. These are the cases where the purpose of the study is to analyze the agents and the society of agents without much regard to optimizing an outcome and often there is no obvious and well-defined utility function to optimize anyways. 


The study of norms and normative systems is one domain where these approaches are far more adopted, domains where analysing overall emergent behaviors, general trajectory of the society of agents, and the effect of norms and policies on that society are the main point of the study. Then the expressivity and transparency of the models and their decision-making becomes more important in analyzing and explaining (non-)compliant behaviors that goes beyond utility functions~\cite{van2019governmental}. 

This research intrinsically falls into this category, and, as it will be presented in later chapters, the Belief-Desire-Intention (BDI) model of agency \cite{Rao1995} is utilized in modelling agents. By specifying agents in terms of human-related attitudes, BDI agents are suitable in modelling social agents. More interestingly, majority of the concepts defined in the previous section about norms, like presence of behavioral specifications that can be matched and concretized in certain situations, or having contingency plans for failures are all already (partially) identified by BDI models to various degrees of success.


\section{Motivation and Research Questions}
%Prior to addressing how a complex data-sharing infrastructure and its participants can be modelled, we need to address why modelling such systems is important. 
The overarching motivation of this research has been in creation of a digital data market-places (DMP) specially in the field of logistics data. The need for data is exponentially increasing in all research and industries, and an environment like a DMP can provide different parties a place to share (or buy and sell) scientific or corporate data with each other. Although the presence of such environment can vastly improve the efficiency of data-oriented research and even prevent issues like data monopolies, there are certain risks involved, such as privacy issues, security issues, competitive corporate advantages, legal and ethical challenges, and, governance and accountability concerns.

Furthermore, as more governments (and other authoritative bodies) are implementing regulations to govern data transactions, such sharing environment needs to be compliant with these regulations, also actors in these markets like organizations and companies may have ad-hoc contractual agreements about data-sharing, and, they may also have internal policies like user agreements about how they can share data. The complexities of the market-places alongside the requirement for compliance results in a need for appropriate governance.

A crucial part of governance are policies. They provide a framework for decision-making, establishing guidelines and rules, and guiding the actions of individuals and organizations. Policies are generally about setting directions and goals for a system, regulating behavior, managing resources and risks, and, promote accountability. In the domain of data-sharing, policies have more specific roles.
%This work started by focusing on exactly that, how do we govern a data-sharing infrastructure and how can we create the policies for them? 

From the perspective of privacy, policies may specify how personal and sensitive data should be handled, defining the conditions for data anonymization and de-identification. Policies can also ensure compliance with relevant laws and regulations related to data protection, intellectual property, consent, and confidentiality. They define appropriate criteria for data access, such as applicable intentions and performance of appropriate actions, like obtaining consent. 

Policies can ensure fair and equitable data sharing, and establish rules for data usage, reuse, and retention~\cite{nissenbaum2004privacy}. They address ethical considerations such as fairness, transparency, and accountability in data-sharing practices. Policies may also be about data security, they establish measures to protect data from unauthorized access, breaches, and cyber threats. They outline security protocols, encryption standards, access controls, sufficient monitoring and data breach notification procedures. 

Another important aspect of policies and policy-making involves engaging relevant stakeholders, including individuals, industry representatives, researchers, and policymakers, to gather input, address concerns, and ensure that diverse perspectives are considered in shaping the data-sharing policies.


The importance of policies in data-sharing intuitively results in the crucial role of policy-making. In data-sharing, policy-making generally refers to the process of developing and implementing rules, guidelines, and frameworks that govern the collection, storage, access, use, and sharing of data and many other aspects of data-sharing. It involves the creation of policies, regulations, and standards that define the rights, duties and responsibilities, permissions, obligations, and power relations of various stakeholders involved in data-sharing, such as individuals, organizations, and governments. This lead to the main goal of this dissertation: \textit{Defining approaches, methodologies and tools for policy-making in the data-sharing domain}.

Policy-makers are required to take into account the effect and impact of their policies, for long-term and complex policy matters, this is only feasible with mathematical or computer-based models~\cite{Boulanger2005}. Modelling can play a crucial role in policy-making by providing insights, predictions, and evaluations that inform the development, implementation, and assessment of policies. Modelling helps with understanding complex systems allowing policymakers to gain a better understanding of systems and their dynamics. Modelling enables scenario analysis by simulating different scenarios and predict the potential outcomes of policy choices resulting in more optimized policies with more balanced between goals. Finally, models can facilitate more transparency and better policy communication with stakeholders and even system designers.

There are different types of modelling used in the different communities to assist in policy-making. These approaches vastly differ in their focus, methodology, and, modelling time-frame. From macroeconomics to system dynamics to agent-based modelling, each approach has its own use cases in specific domains, and there are arguments to use one or a combination of these approaches in different use cases~\cite{Scholl2001,silverman2023,dosi2019,Dignum2008}. Agent-based models are suitable for policy-making as they define the behavior of individuals to build emergent trajectory of the system as a whole. In a real system, where policies are based on top-down assumptions of behavior, many changes occur bottom-up from individual actors' actions~\cite{Dignum2008}. This dissertation focuses on utilizing agent-based models for policy-making; then, the more concrete goal of this research would be: \textit{Defining approaches, methodologies and tools for policy-making in the data-sharing domain based on Agent-Based Modelling}.



While this dissertation focuses on data-sharing, it is not only data-sharing use-cases that are in need of such methods for policy-making and system design. More aspects of our day-to-day life are being \textit{controlled} by automated processes; visa applications, job applications, credit placements, mortgage and insurance are just a few examples and one can only imagine that the list will continue to grow as time goes on. Even if data-sharing regulations are a relatively new phenomena, when taken in a broader sense, there are already regulations implemented for other domains that are now becoming more automated, and when the decisions are made in an automated manner, then the goal will expand to defining policies for any cyber-infrastructural system with respect to arbitrary (regulative and constitutive). This dissertation takes the aforementioned goals, by assuming (agent-based) modelling and model execution as a principal step in system design and policy-making:

\begin{displayquote}
\textbf{Main Research Question:} \textit{How can we model a norm-governed cyber-infrastructure for the purpose of policy-making?}
\end{displayquote}


\section{Approach and Scope of this Thesis}
The first step in defining the scope of this work is to break the main research question structurally by further defining what needs to be modelled: a norm-governed cyber-infrastructural system. The main components are norms, social setting and software/infrastructure that create the system as a whole. We also need to define what are the requirements of these models to be suitable for policy-making.
%but also there is the assumption of openness, which means the system that is being modelled interacts with, affects and get affected by an external undefined environment. 
Then, the main research sub-questions should intuitively become how to model each of these three components. But, unfortunately it is not that simple as these concepts are not at the same level of abstraction for the purposes of this research. 
%Starting from the concept of social setting, it is arguably the main focus of this work, modelling social agents seems to be one of the keys to modelling such systems, but why? 

Take the concept of norms, while modelling them is essential for the whole picture, it is still social agents that act as the governance bodies that regulate other entities of the system, and in effect, it is social agents that are being governed, i.e., it is very hard to talk about norm-related concepts (e.g. sanction) without referring to human-related concepts (e.g. intention). Indeed it is rare to see an infrastructure being punished for non-compliant actions without referring to a social agent (a person, company, institution, organization) as the main liable entity for those actions, or, it would be untenable to say a piece of software or infrastructure is imposing a sanction without the presence of a social agent (an organization, consortium, government)  holding the power to impose that punishment. In effect, norms as are studied in this research ---as it is often the case in real life--- do not have isolated meaning by themselves and without the social setting. The same logic goes for software and infrastructural components, while being another important part of the system model, it is still their usage by and effects on social agents that is being scrutinized.
%Furthermore, while software modelling is a mature and advanced field of study, the purposes of this work are aimed at possibility of a more realistic view of software systems. Such realistic view reveals that software and infrastructural systems are very technology-dependant subjects and high-level models of them are rarely complete without their low-level technological details. It is hard to discuss about a data transfer without mentioning if that data is personal or not, anonymized or not, plain or encrypted, or if the transmission channel is secure or not. 

In summary, in the context of this dissertation, social agents are the glue that hold the models together and they have the following requirements for them to be suitable in policy-making: (1) Effective social simulation that is suitable for for policy makers requires on the ability to model the individual decision making process given subjective social norms, individual preferences, and policies, in other words, highly expressive agents in terms of cognitive capabilities. (2) The agent models should have high scalability; this is important in this context to grant the model designers the freedom to create models with a high number of agents and interactions.; (3) The framework should be modular. Modularity is crucial because there are many different theories in the literature about norms, preferences, personal policies, and other aspects of the agents. A highly modular agent architecture allows us to experiment with these theories without the need to hard-code them into the reasoning cycle of the agent. This gets us to the first research question:



\begin{displayquote}
\textbf{Research Question 1:} \textit{How to create expressive, scalable and modular models of social agents?}
\end{displayquote}


%The short answer to this question is that agent-based programming is the most suitable approach to create models of agents. The alternate approach that is mathematical models while having many advantages are simply not transparent enough at the required level of abstraction for our use-cases: it is not easy to model intention for example in an effectively optimizer agent that has a utility function driving its decisions. 

 Belief-Desire-Intention model of agency has been identified in the literature as a suitable approach to create software agents with the ability to reason about norms~\cite{Dignum2002MotivationalNorms,Deljoo2018APlaces}, have preferences~\cite{Visser2016}, and can be utilized effectively in policy-making~\cite{dignum2008towardsagents}. While the reasons behind this choice will be thoroughly explored through this work, the approach adopted to agent-based programming is the BDI model to create the social agents. Although there are multiple BDI frameworks introduced in the literature, after much deliberation and testing presented in Chapter~\ref{ch:asc2}, it turned out they did not meet the requirements of scalabilty and modularity, and as it is later discussed interoperability.


In the process of this work, a BDI (and MAS) framework called AgentScript Cross-Compiler (ASC2) was designed and developed mainly with the scalability and modularity requirements in mind which uses a programming language based on the AgentSpeak(L). ASC2's design is highly modular, as it does not assume a hard-coded reasoning and decision-making cycle for agents, in fact, almost every aspect of the agents is programmable. 

Firstly ASC2 utilizes actor-oriented programming via Akka actor framework meaning each agent consists of multiple actors each with their own role that can communicate through internal messaging, effectively making an agent a modular actor micro-system in itself. Secondly, by following software engineering best practices and methodologies like Dependency Injection and Inversion of Control. By using Dependency Injection, most components and sub-components of an an agent can be sent to it as potentially customized dependencies. Furthermore, with Inversion of Control it is the lower level components that are mainly controlling the nuanced execution cycle when higher level components only define abstract control flows. These design decisions also result in scalability of the framework, the ASC2 framework is introduced, analysed, and, benchmarked from an engineering perspective in Chapter \ref{ch:asc2}. 

Next, there is the issue of how to model software and infrastructural entities. Software and infrastructure components are one of the main components of our models and being able to include them is essential. However, the question of how to model the software components falls out of the scope this thesis as there is an extensive body of research on software modelling Instead, the focus is on integrating software components with the software agents. Steps are taken to provide a high level of interoperability for social agents modelled in ASC2 to virtually any type of model of software and infrastructural entities that are being used by the designer and policy-maker, including the actual real entities.

\begin{displayquote}
\textbf{Research Question 2:} \textit{How can social agents utilize software and infrastructural models or entities?}
\end{displayquote}

One of the bigger drawback of existing BDI frameworks that was identified in the context of this research, which was also a motivation in creating ASC2, is lack of straightforward way for agents to interact with arbitrary environments. Here, environment means a arbitrary piece of software external to the agent. This can be a communication interface, an execution environment, or any other type of software component. In the data-sharing context for example, this environment is the data-sharing infrastructure --- or a model of it-- that the agents need to communicate with through some arbitrary API. Interestingly, this is hardly recognized in the classic MAS literature as a challenge or requirement, and therefore, the lack of it as a drawback. However, by reviewing the more recent literature we can observe that there are multiple works~\cite{Rafalimanana@2020,Collier2019,Mitrovic@2013} that try to interface existing BDI frameworks with modern software components and architectures like web services or micro-services.

ASC2's design however gives the agents the advantage of simple interoperability with the external environment. As it is addressed in Chapters~\ref{ch:devops} and~\ref{ch:normative_advisors}, after compilation ASC2 agents are technically JVM programs meaning they are indeed interoperable with any system that for example a Java program can utilize or communicate with. This results in much more effective modelling cycles as there is far less concern about connecting the models to a real system or interfaces.


Next is the notion of norms, modelling them including social norms, regulations, contractual agreements, internal policies is an important part of the modelling that this thesis aims at. However, while different approaches and ideas about modelling norms are explored in this work, defining any novel approach to do so falls out of the scope of this thesis for both theoretical and practical reasons. The practical reason is mainly that other threads of dedicated research were being performed on this matter within the research group that both affected and utilized this work. The theoretical reason is that there are various levels of abstraction that norms can be modelled for different use-cases and this work stays agnostic to which approach is the most suitable. However, the question that is studied is the concept of norms from the perspective of social agents:

\begin{displayquote}
\textbf{Research Question 3:} \textit{How can social agents reason with, and, about norms?}
\end{displayquote}

There are multiple works in the literature about the interaction of agents with norms~\cite{{Dignum2002MotivationalNorms,Broersen01theboid,Tufis2017,Criado2010TowardsCompliance}}. The main concern of this dissertation on this question is to identify the interactions between social agents and norms, including the interpretation and qualification of events between institutional and physical realities, understanding compliant and non-compliant behaviors and the relation between compliance and autonomy. This is addressed in section \ref{ch:eumas} with introducing the concept of normative advisors, a flexible approach that can be utilized by social agents to have an understanding of normative positions in their environment without reducing their autonomy.


%This thesis started from the assumption that agent-based programming is a more expressive approach than mathematical models with utility functions. 
%However, it also revisits the validity of that assumption. 
Modelling agents for the purpose of policy-making requires to reproduce to a certain extent decision-making constructs similar to those observed in human institutions. Furthermore, for \textit{traceability} and \textit{explainability} reasons, decision-making that precedes actions is as relevant as the behavior. Although BDI models --- and more specific to this work, AgentSpeak(L)-like models--- are designed with traceability and explainability as a first class requirements~\cite{RaoAS1996}, when it comes to more complex forms of decision-making -- like the issues that manifest in conflict resolution-- the question of modelling objectives, desires, and preferences of agents are not trivial issues~\cite{Dignum2002MotivationalNorms,Visser2016,Mohajeri2020}.  
%Mathematical models of agents are mainly aimed at optimizing a certain utility function, it can be conceptually viewed as if such an agent has the quantitative objectives, desire or preference of maximizing that function. 

In the process of this research it was observed that by adding norms and normative reasoning into the agents, these concepts become even more important. Norms in every form can be conflicting and such conflicts can rarely be resolved by typical approaches~\cite{Boella2014,zurek2016fedcsis}. Furthermore, norms can also be conflicting with the goals that an agent is trying to achieve~\cite{Broersen2001ResolvingDesires}, how should an agent program behave in face of conflicts? on what basis should it make decisions in such situations? There is not clear cut answer to these questions as it will depend on the context. This dissertation addresses this by taking the stance that it depends on the preferences and desires of the agent. It is the agent that should decide based on the context how to resolve a conflict and the designer should be able to program such notions. The issue addressed here is not about what is the \textit{most appropriate} behavior of the agent as the consequence of its decision-making and conflict resolution; that will depend on what the designer is modelling. Instead, it is how such decision-making can be expressed in a traceable manner:

\begin{displayquote}
\textbf{Research Question 4:} \textit{How can we make the agents' decision-making traceable and explainable?}
\end{displayquote}

This question is addressed in Chapter \ref{ch:aamas} by adding explicit preferences in the form of CP-Nets into the BDI model and the AgentScript programming language. With this approach, the designer can separate the concern between the procedural or the \textit{how-to} knowledge of the agent program from the preferential or \textit{what-to} knowledge. This makes the decision-making of the agents more transparent which is a requirement in models that are created for the purpose of policy analysis. 

The last issue addressed in this thesis is the practicality of utilizing agent-based models as part of real-world system design and policy-making, not only as they are used in this work but the whole agent-based modelling community. It is always the case that accessibility and usability of the tools in a certain methodology is an important part of their adoption, it is hard and mostly not feasible to convince domain experts like programmers to utilize an approach if they need to also learn a whole new set of tools and ecosystems. This is also the case for utilizing agent-based models and has been a major concern in this work

\begin{displayquote}
\textbf{Research Question 5:} \textit{How to make agent-based modelling a usable approach for mainstream designers?}
\end{displayquote}

In the recent years the tools created for design and development in mainstream software community are becoming more advanced and efficient. A few examples of such tools are IDEs, testing libraries, build tools, code coverage tools, code repositories, and, DevOps system like CI/CD tools. Intuitively it is advantageous to allow for utilization of these tools in agent-based modelling and model execution. As a side effect of its design, ASC2 programs can directly utilize any system or library available to any other software programs without the need for any extra\footnote{Extra meaning more than any other piece of software} interfacing as after compilation, ASC2 programs become JVM-based programs. This includes all the development tools listed above and many others. In Chapter \ref{ch:devops} it is shown how utilizing these tools, not only benefits agent and multi-agent system (MAS) communities in e.g., testing their designs with mainstream automated testing tools, but more importantly, allows agent models to be used as part any software development process, be it for testing or any other purpose.


\section{Research Context and Collaborations}
Before starting the next chapter, I will try to put this dissertation into the context and environment that the research was conducted, highlighting the parallel research that had a connection to it and some of the events that affected it. The research was done in the Complex Cyber-Infrastructures (CCI)\footnote{\url{https://cci-research.nl/}} research group and was funded by the project Data Logistics for Logistics Data (DL4LD)\footnote{\url{https://dl4ld.nl/}}. There were two other closely related projects: Secure Scalable Policy-enforced Distributed Data Processing (SSPDDP)\footnote{\url{https://cr-marcstevens.gitlab.io/sspddp/}} and Enabling Personalized Interventions (EPI)\footnote{\url{https://enablingpersonalizedinterventions.nl/}} that shared research with DL4LD.

Following highlights some of the parallel research threads that were conducted mostly by other PhD candidates and post-doctoral researchers alongside this work. In the context of this issertation, Agent-Oriented Programming approaches were utilized for developing Agent-Based Models. The counterpart to this are the mathematical models of agents which were simultaneously studied in the context of the group by Fractic et. al. \cite{Peter}. Where this thesis tries to find approaches in enhancing models for developing policies, there were parallel works that focused on what these policies should be, including risk management and enforcement schemes by Zhou et. al. \cite{Xin}. Another thread of research, as it was mentioned in the introduction was about norm reasoning. The results of these works done by Thomas, Lu-Chi, Milen \cite{Thomas,Milen,Lu-Chi}, are used throughout future chapters . Finally there was a more network-focused side of this work, mainly on development of Data Market-Places e.g., Reggie et. al. \cite{Reggie}, Lu et. al. \cite{Lu}.

Finally, majority of the time dedicated to this research overlapped with COVID-19 restrictions, and apart from ``normal'' issues, it meant there was little to no real interaction with the industry partners that were the initial stakeholders of this project. To provide full transparency for the reader the phrase ``it turned out that it is not only data-sharing use-cases that are in need of such methods for policy-making and system design'' in the motivation section, while still true, also partially means that it is considerably more challenging to study policies and policy-making in data-sharing without the presence of stakeholders that have access to data and are interested in sharing them.  As a result, while this research was at a relatively high level of abstraction at its inception, for better or worse, became even more theoretical that it was intended to be ---or I intended it to be---, but fortunately, my supervisors were already interested in the more theoretical side of the issue \cite{Tom,Giovanni} which greatly guided this research in every aspect. However, by interacting with the respective communities e.g., AI and Law, Normative Systems, and, Multi-Agent Systems, my observation was that there is an interest for more practical and usable approaches to bring the long standing results of these communities closer to mainstream domains. For this reason, every chapter of this dissertation includes development of tools or proof of concepts that utilize fairly mainstream software ecosystems.  

\section{Structure of the Dissertation}



