\chapter{Conclusion and Future Works}

%The aim of this thesis is to explore approaches in modelling norm-governed cyber-infrastructures. Modelling such systems in itself, can be done via multiple approaches that are favored in different research communities. In this work however, the focus was on utilizing agent-based programming to model the system as a collection of parties that are individually modelled as either intentional agents or software/hardware infrastructural components.

In this Chapter, firstly the motivations that drive this research are reiterated. Then, the work presented throughout the previous chapters is summarized. Next, the achievements of the dissertation are concluded by assessing the present work against the research questions presented in the introduction whilst identifying the limitations. Finally, the possible future research directions are highlighted.

\section{Motivation}
Governance refers to the processes, and structures through which both organizations, societies as a whole, and actors within them are controlled and regulated. Governance includes mechanisms by which decisions are made, normative relations are defined, authority is exercised, and actions are taken to achieve goals and fulfill responsibilities and expectations.

Policies and policy-making play a crucial role in governance. Policies provide a set of rules, regulations, and procedures for actors that guides decision-making and behavior. They outline the objectives, values, and behavioral patterns that need to be followed to achieve desired outcomes and promote consistency, coherence, and, effective coordination individual actors' behavior. Policies can cover various areas such as legal compliance, ethical standards, operational procedures, resource allocation, and risk management.

Policy-making in the data-sharing domain, or the broader software domain is becoming increasingly important. As software systems are getting more involved and integrated in the societies, there is an increasing need for governance over them to make sure that these systems are regulated in accordance with the societal norms,  furthermore there is a need for approaches to make effective policies for them. However, there are many issues that challenge governing software (AI or otherwise) systems \cite{}. As there are many recent example, such systems generally grow fast, have little transparency, and affect the society in many ways, meaning newer tools are needed for both policy-makers and system designers to allow them to address these issues. This dissertation focuses on this issue, exploring challenges and approaches for efficient policy-making for systems that include both social and software actors and are governed by a set of regulations.

Modelling is one of the tools utilized in policy-making. Models can give insight about a system and assist in predicting the trajectory of that system in the future. Agent-based modelling in particular is a powerful tool for policy-making as such models start from defining the behavior of individual actors to infer the high-level emergent behavior of the system as a whole. This is the approach utilized in this dissertation, utilizing system and agent-based modelling as a well-known policy making tool.


\section{Summary}

This dissertation can be separated in two main parts, after the introduction, Chapters~\ref{ch:asc2} to \ref{ch:devops} mainly revolve around modelling social agents as BDI agent scripts, putting an emphasis on usability of these models through enhancing scalability, transparency and testability of the models. The second part in Chapters~\ref{ch:normative_advisors} to \ref{ch:jurix} are about modelling social norms, specifically the interaction of social agents with norms to complete the policy-making aspect of this dissertation. 

In Chapter~\ref{ch:asc2} an agent-based programming framework called AgentScript Cross-Compiler ASC2 is introduced. This framework is based on the Belief-Desire-Intention model of agency which is utilized in the literature to model a wide range of socio-technical systems~\cite{...}. ASC2 uses a language based on AgentSpeak(L)/Jason~\cite{Jason}. While there are multiple well-developed BDI-based frameworks exist in the literature, the main requirements and advantages of ASC2's design are scalability and usability. 

In summary, AS2 is a cross-compiler that takes as input agent programs developed in a high-level language and translates them to executable JVM-based programs. This makes ASC2 models virtually as scalable as any JVM program. Furthermore, the frameworks utilizes actor-model (implemented in Akka), giving it a consistent and robust backbone for its concurrency. ASC2's language, albeit heavily influenced by previous works, has unique characteristics, most importantly, JVM languages (Java/Scala/Kotlin) statements are allowed as part of the language, making it a polyglot between a reactive agent language, a prolog-like logic language, and a JVM-like language. From a performance perspective, Chapter~\ref{ch:asc2} presents multiple benchmarks to compare ASC2 with other frameworks which shows ASC2 performs better or equal to other existing frameworks.

Chapter \ref{ch:preferences} explores the idea of adding explicit preferences at the language-level to BDI agent; utilizing them to allow the agent to refine abstract goals at run-time based on the situation to achieve its desires. Where BDI scripts are typically about \textit{what-to} knowledge of an agent, Chapter \ref{ch:preferences} illustrates that adding preferences enhances them with \textit{how-to} knowledge: given a partially abstract goal, the agent can gradually concertize and achieve that goals based on the preference the context of the environment that the agent resides in. By allowing explicit preferences to an agent script, many decisions that could be opaque as they happen within the reasoning engine of the agent, become also explicit and transparent. 

To model the preferences, ASC2 utilizes CP-nets, short for ''Conditional Preference Networks``, a type of graphical model used to represent and reason about preferences. They are commonly used in the field of artificial intelligence, specifically in preference modeling and decision making. Chapter \ref{ch:preferences} also illustrates a novel form to represent CP-Nets alongside and algorithm to find optimal decisions in Prolog, proves the correctness of the algorithm and shows the time complexity of the the algorithm is on-par with what is presented in \cite{Boutilier2004}. Furthermore, this new form can also be used to express contextually conditioned and parameterized preferences, resulting in more flexibility than pure CP-Theories. To reiterate an example, one can define a preference statement such as ``I prefer the place I am already at,'' that depends completely on the state of the agent in the environment; this makes CP-Nets, a model that is classically used for static one-time decisions in a fully known environment suitable for agents acting in dynamic environments with limited information.


Chapter~\ref{ch:devops} studies testing and verification of agent-based models at scale and in real world settings. As software engineering community already has many advanced and mature tools for testing, Chapter~\ref{ch:devops} aims as creating interoperability between agent-based modelling frameworks with mainstream software development tools such as automated testing and integration. It is illustrates the interfacing of ASC2 with multiple tools such as build tools, unit and integration testing,  continuous integration and deployment, and, code coverage systems. It is further explored to present requirements that would allow other frameworks adopt the same idea.

Chapter~\ref{ch:normative_advisors} introduces a flexible agent architecture for introducing norms in multi-agent systems. The basis of this architecture are normative advisors. In short, a normative advisor is a normal BDI agent, except that its inference engine --which is typically a logic-based reasoning engine-- with a norm reasoning framework. With this approach, these advisor agents can be utilized by other agents as an external source to maintain and reason about an institutional perspective (state) based on an specific set of norms. The advisors can be initialized with a particular norm specification and maintain an institutional perspective on the environment, which is continuously updated at run-time and can be queried at anytime. Both regulative and constitutive norms are taken into account. This chapter creates the basis to allow an agent-based model being utilized in policy-making.


Chapter~\ref{ch:cmf} and \ref{ch:jurix} are two illustrative examples of how the proposed approaches and tools in this dissertation can be utilized to model a normative system. Chapter~\ref{ch:cmf} utilizes ASC2 and its normative extension to illustrate, implement and analyze a model of a Data Market-Place as a normative multi-agent system. Norms in this chapter are used as a coordination mechanism between participants of the market to assist policy-makers by providing insights into the effect of policies and even generating policy and system design artifacts. The market participants in this example have use ad-hoc contractual agreements and their bounded view of the events in the system to create their own institutional view of the market, by doing so, each participant creates a dynamic set of expectations about what shall happen in the system. This means based on the pre-agreed norms and event happening in real time, each actor infers their duties and claim towards other participant and act based on this knowledge. 

Chapter~\ref{ch:jurix} is another illustrative example, focusing on other challenges in modelling normative systems, namely mixing qualitative, quantitative and normative reasoning. The agents in this chapter act based on external norms that they have adopted and internal preferences, both of which depend on the quantitative state of the environment. While the scenarios of this chapter are quite simple, they illustrate how a combination of different modelling approaches in one framework can allow for more expressive models.



\section{Research Results}

As described in Chapter~\ref{ch:intro}, the main goal of this dissertation is \textit{defining approaches, methodologies and tools for policy-making in the data-sharing domain}. The focus in this research is to utilize modelling, particularly agent-based models for policy-making. The main research question being: 
\begin{displayquote}
    \textit{How can we model a norm-governed cyber-infrastructural system for the purpose of policy-making?}
\end{displayquote}

To address this question, this work explores what are the different components of these models that need to be modelled and what are the requirements of these models to make them suitable for policy-making specifically in data-sharing and presents a set of tools in form of an agent programming framework. To identify and build up the components of this framework, this problem is broken down into a set of research questions that the work presented in this dissertation is aimed to answer:

\begin{displayquote}
\textbf{Research Question 1:} \textit{How to create expressive and modular models of social agents?}
\end{displayquote}






\textbf{Research Question 2:} \textit{How can social agents utilize software and infrastructural models or entities?}

\textbf{Research Question 3:} \textit{How can social agents reason with and about norms?}

\textbf{Research Question 4:} \textit{How can we model desires and preferences of agent?}

\textbf{Research Question 5:} \textit{How to make agent-based modelling a usable approach for policy-makers and designers?}



\section{Limitations}

\textbf{still some computaionl limitations specially in logic and norm reasoning}

\textbf{}

\section{Future Works}