\chapter{Conclusions and Future Work}

%The aim of this thesis is to explore approaches in modelling norm-governed cyber-infrastructures. Modelling such systems in itself, can be done via multiple approaches that are favored in different research communities. In this work however, the focus was on utilizing agent-based programming to model the system as a collection of parties that are individually modelled as either intentional agents or software/hardware infrastructural components.

In this Chapter, firstly the motivations that drive this research are reiterated. Then, the work presented throughout the previous chapters is summarized. Next, the achievements of the dissertation are concluded by assessing the present work against the research questions presented in the introduction whilst identifying the limitations. Finally, the possible future research directions are highlighted.

\section{Motivation}
Governance refers to the processes, and structures through which both organizations, societies as a whole, and actors within them are controlled and regulated. Governance includes mechanisms by which decisions are made, normative relations are defined, authority is exercised, and actions are taken to achieve goals and fulfill responsibilities and expectations.

Policies and policy-making play a crucial role in governance. Policies provide a set of rules, regulations, and procedures for actors that guide decision-making and behavior. They outline the objectives, values, and behavioral patterns that need to be followed to achieve desired outcomes and promote consistency, coherence, and, effective coordination individual actors' behavior. Policies can cover various areas such as legal compliance, ethical standards, operational procedures, resource allocation, and risk management.

Policy-making in the data-sharing domain, or the broader software domain is becoming increasingly important. As software systems are getting more involved and integrated in the societies, there is an increasing need for governance over them to make sure that these systems are regulated in accordance with the societal norms,  furthermore there is a need for approaches to make effective policies for them. However, there are many issues that challenge governing software systems including AI~\cite{}. The number of AI system has rapidly increased over the last couple of years and lack of transparency, concerns about privacy and other legal and ethical issues have increased the need for effective policies to control AI and more generally IT-infrastructures including data-sharing.
This dissertation focuses on this issue, exploring challenges and approaches for such efficient policy-making for systems that include both social and software actors and are governed by a set of regulations.

Modelling is one of the tools utilized in policy-making. Models can give insight about a system and assist in predicting the trajectory of that system in the future. Agent-based modelling in particular is a powerful tool for policy-making as such models start from defining the behavior of individual actors to infer the high-level emergent behavior of the system as a whole. This is the approach utilized in this dissertation, utilizing system and agent-based modelling as a well-known policy making tool.


\section{Summary}

This dissertation can be separated in two main parts, after the introduction, Chapters~\ref{ch:asc2} to \ref{ch:devops} mainly revolve around modelling social agents as BDI agent scripts, putting an emphasis on usability of these models through enhancing scalability, transparency and testability of the models. The second part in Chapters~\ref{ch:normative_advisors} to \ref{ch:jurix} are about modelling social norms, specifically the interaction of social agents with norms to complete the policy-making aspect of this dissertation. 

In Chapter~\ref{ch:asc2} an agent-based programming framework called AgentScript Cross-Compiler ASC2 is introduced. This framework is based on the Belief-Desire-Intention model of agency which is utilized in the literature to model a wide range of socio-technical systems~\cite{...}. ASC2 uses a language based on AgentSpeak(L)/Jason~\cite{Jason}. While there are multiple well-developed BDI-based frameworks described in literature, the main requirements and advantages of ASC2's design are scalability and usability. 

In summary, AS2 is a cross-compiler that takes as input agent programs developed in a high-level language and translates them to executable JVM-based programs. This makes ASC2 models virtually as scalable as any JVM program. Furthermore, the frameworks utilizes actor-model (implemented in Akka), giving it a consistent and robust backbone for its concurrency. ASC2's language, albeit heavily influenced by previous works, has unique characteristics, most importantly, JVM languages (Java/Scala/Kotlin) statements are allowed as part of the language, making it a polyglot of a reactive agent language, a prolog-like logic language, and a JVM-like language. From a performance perspective, Chapter~\ref{ch:asc2} presents multiple benchmarks to compare ASC2 with other frameworks which shows ASC2 performs better or equal to other existing frameworks.

Chapter \ref{ch:preferences} explores the idea of adding explicit preferences at the language-level to BDI agent; utilizing them to allow the agent to refine abstract goals at run-time based on the situation to achieve its desires. Where BDI scripts are typically about \textit{what-to} knowledge of an agent, Chapter \ref{ch:preferences} illustrates that adding preferences enhances them with \textit{how-to} knowledge: given a partially abstract goal, the agent can gradually concretize and achieve that goals based on the preference the context of the environment that the agent resides in. By allowing explicit preferences to an agent script, many decisions that could be opaque as they happen within the reasoning engine of the agent, become also explicit and transparent. 

To model the preferences, ASC2 utilizes CP-nets, short for ''Conditional Preference Networks``, a type of graphical model used to represent and reason about preferences. They are commonly used in the field of artificial intelligence, specifically in preference modeling and decision making. Chapter \ref{ch:preferences} also illustrates a novel form to represent CP-Nets alongside an algorithm to find optimal decisions in Prolog, proves the correctness of the algorithm and shows that the time complexity of the algorithm is on-par with what is presented in \cite{Boutilier2004}. Furthermore, this new form can also be used to express contextually conditioned and parameterized preferences, resulting in more flexibility than pure CP-Theories. To reiterate an example, one can define a preference statement such as ``I prefer the place I am already at,'' that depends completely on the state of the agent in the environment; this makes CP-Nets, a model that is classically used for static one-time decisions in a fully known environment suitable for agents acting in dynamic environments with limited information.


Chapter~\ref{ch:devops} studies testing and verification of agent-based models at scale and in real world settings. As the software engineering community already has many advanced and mature tools for testing, Chapter~\ref{ch:devops} aims at creating interoperability between agent-based modelling frameworks with mainstream software development tools such as automated testing and integration. It illustrates the interfacing of ASC2 with multiple tools such as build tools, unit and integration testing,  continuous integration and deployment, and code coverage systems. It furthermore explores how we could present requirements that would allow other frameworks adopt the same idea.

Chapter~\ref{ch:normative_advisors} introduces a flexible agent architecture for introducing norms in multi-agent systems. The basis of this architecture are normative advisors. In short, a normative advisor is a normal BDI agent, except for that its inference engine --which is typically a logic-based reasoning engine-- can reason with and about norms. With this approach, these advisor agents can be utilized by other agents as an external source to maintain and reason about an institutional perspective (state) based on a specific set of norms. The advisors can be initialized with a particular norm specification and maintain an institutional perspective on the environment, which is continuously updated at run-time and can be queried at anytime. Both regulative and constitutive norms are taken into account. This chapter creates the basis to allow an agent-based model being utilized in policy-making.


Chapter~\ref{ch:cmf} and \ref{ch:jurix} presents two illustrative examples of how the proposed approaches and tools in this dissertation can be utilized to model a normative system. Chapter~\ref{ch:cmf} utilizes ASC2 and its normative extension to illustrate, implement and analyze a model of a Data Market-Place as a normative multi-agent system. Norms in this chapter are used as a coordination mechanism between participants of the market to assist policy-makers by providing insights into the effect of policies and even generating policy and system design artifacts. The market participants in this example use ad-hoc contractual agreements and their bounded view of the events in the system to create their own institutional view of the market, by doing so, each participant creates a dynamic set of expectations about what shall happen in the system. This means based on the pre-agreed norms and events happening in real time, each actor infers their duties and claim towards other participants and act based on this knowledge. 

Chapter~\ref{ch:jurix} presents another illustrative example, focusing on particular challenges in modelling normative systems, namely mixing qualitative, quantitative and normative reasoning. The agents in the example described in is chapter act based on external norms that they have adopted and internal preferences, both of which depend on the quantitative state of the environment. While the scenarios used in the example described in this chapter are quite simple, they illustrate how a combination of different modelling approaches in one framework can allow for more expressive models.



\section{Research Results}

As described in Chapter~\ref{ch:intro}, the main goal of this dissertation is \textit{defining approaches, methodologies and tools for policy-making in the data-sharing domain}. Modelling plays an important role in policy-making; scenario simulation via model execution can give insight and possible predictions about the trajectory of the system. Furthermore, models facilitate stakeholder engagement and transparency about the system and even guide the development process. Among the different types of modelling, Agent-based models are specially suitable for policy-
making as they define the behavior of individuals to build emergent behavior of the system as a whole~\cite{dignum2008towardsagents}. The focus in this research is to utilize modelling, particularly agent-based models for policy-making. The main research question being: 
\begin{displayquote}
    \textit{How can we model a norm-governed cyber-infrastructural system for the purpose of policy-making?}
\end{displayquote}

To address this question, this dissertation explored what are the different components of these models that need to be modelled and what are the requirements of these models to make them suitable for policy-making specifically in data-sharing and presented a set of tools in form of an agent programming framework called ASC2. To identify and build up the components of this framework, this problem was broken down into a set of research questions that the work presented in this dissertation aimed to answer:

\begin{displayquote}
\textbf{Research Question 1:} \textit{How to create expressive, scalable and modular models of social agents}
\end{displayquote}

Although there are many requirements that agent models need to satisfy, expressivity, scalability, and, modularity are identified to have higher priority in the context of this research. 

\paragraph{Expressive Power} The main motivation of using agent-based models in policy-making is to build-up the system level behavior from agent behavior to analyse the effect and impact of policies, this requires  modelling the individual decision making process given subjective social norms, individual preferences, objectives, goals and desires. Creating such models demands highly expressive agent models in terms of cognitive capabilities. In Chapter~\ref{ch:asc2} the ASC2 agent-programming framework is introduced. ASC2 is based on the BDI model of agency~\cite{Rao1995} utilizing an agent programming language called AgentScript which is based on the AgentSpeak(L)~\cite{RaoAS1996}. BDI's agent theory in this dissertation refers to Bratman's theory of practical reasoning \cite{bratman1987intention}, describing the agent's cognitive state and reasoning process in terms of its \textit{beliefs}, \textit{desires} and \textit{intentions}, attributes usually used to describe human behaviors, making them highly expressive in modelling social agents~\cite{Fisher2007}. Apart from intrinsic attributes of BDI, ASC2 takes extra steps in adding preferences at the language level and abstract goal refinement at decision-making level in Chapter~\ref{ch:preferences} and adding norm reasoning at the framework level in Chapter~\ref{ch:normative_advisors}.

\paragraph{Scalability} Often policy-making requires modelling a system with a high number of individuals. This creates a challenge for agent-based modelling, particularly if these models each have a complex cognitive models with higher computational resource demand. Scalability then becomes a first-class requirement as having complex agents is not beneficial in policy-making if we can only have a few of them in any given scenario. Scalability refers to the ability of the framework to \textit{scale up} by adding more computational resources, typically in a distributed manner. At the execution level, ASC2 utilizes actor-oriented programming via Akka actor framework meaning each agent consists of multiple actors each with their own role that can communicate through internal messaging, effectively making an agent a modular actor micro-system in itself. This means that not only between agents, but even within each agent can have internal asynchronous execution of concurrent tasks and potential for distributed deployment, making ASC2 highly scalable.
 
 
\paragraph{Modularity} The importance of agent's cognitive capabilities like preferences, norm reasoning, goals and desires was already discussed, however, each of these aspects are studied in multiple research communities resulting in many different valid and interesting theories. From the experience of this dissertation, especially from a research standpoint it is desirable for a framework to be able to easily embed these theories within the agent to allow for experimentation. ASC2 agent are actor micro-systems, consisting of multiple components that can communicate through internal messaging. This makes it quite easy to modify the framework by simply replacing a component. This is illustrated for example in Chapter~\ref{ch:normative_advisors} where normative advisor agents are created by replacing their logic reasoning engine with a norm reasoning framework.



\begin{displayquote}
\textbf{Research Question 2:} \textit{How can social agents utilize software and infrastructural models or entities?}
\end{displayquote}

Software and infrastructural entities are one of the three main categories of the models needed for modelling a norm-governed socio-technical system, meaning being able to include them plays an essential role. In the context of this dissertation, the focus is on allowing for easy integration of the agent models with arbitrary external software. As it was discussed in Chapter~\ref{ch:intro}, this is not a requirement classically recognized by the BDI literature, hence it is not satisfied by most existing BDI-frameworks. However, recent literature shows the interest in integrating agents into other software eco-systems, such as micro-services~\cite{Collier2019}, web services and service oriented architecture~\cite{Rafalimanana@2020}, and traffic simulators~\cite{baumfalk2019sumo}, and even machine-learning software~\cite{lutzenberger2011bdi}, and many of these works describe this integration as challenging. The experience from this dissertation reinforced this, the challenge of integrating software components with agents limits and demotes experimentation.

ASC2 was designed with interoperability as a first class requirement. The cross-compilation step makes sure that the agent models are compatible with any external software that is compatible with JVM-based programming languages , with minimal effort, which intuitively covers vast majority of software ecosystems. This turned out to be extremely useful, essentially being able to \textit{hook} an agent or a whole MAS into any software setting means easy experimentation set-ups which allows for the models themselves to be the main focus instead of trying to communicate with external software, for example network or traffic simulators \footnote{Although never mentioned in the text, ASC2 agents already come with REST and gRPC interfaces out-of-the-box. For example, after running a MAS instance, there are HTTP/REST end-points like \url{http://host/agent/achieve} that can be queried to communicate with each agent, and vice-versa, the agents can use Java's \footnotecode{ java.net.HttpUrlConnection} package to communicate with other HTTP/REST end-points. As the agent's internal entities (goals, beliefs, preferences) are also Java objects, they can simply be (de-)serialized to/from JSON. These were used in multiple internal experiments, including network and traffic simulators.}. This is thoroughly covered in Chapters~\ref{ch:asc2} and \ref{ch:devops}.


\begin{displayquote}
\textbf{Research Question 3:} \textit{How can social agents reason with and about norms?}   
\end{displayquote}

The concept of norms for the purposes of this dissertation covers a wide range of ideas in creating a so-called normative multi-agent system. Norms can be personal policies of individual agents, contractual agreements that groups of agents can utilize to coordinate their collective goals, high-level societal laws that the actors can take into account in their decision-making, rules dictated to (software) agents to make sure they behave within certain boundaries, or, even  a set of external rules which do not affect the behavior of actors but are used to monitor the behavior of the system as a whole. A normative multi-agent system then can include more than one of the aforementioned types norm implemented in it. Furthermore, while eFLINT norm reasoner is predominantly utilized, this dissertation does not assume any specific type of norm reasoning, this allows for more flexibility and experimentation.

To cover all of these areas, Chapter~\ref{ch:normative_advisors} considers the capability for the agents to have an institutional perspective over their environment by introducing the concept of a norm instance: the institutional state of the environment, built upon a normative source and continuously updated via observations. Furthermore,  Chapter~\ref{ch:normative_advisors} proposes a set requirements for social agents based on~\cite{Boella2006IntroductionSystems} to allow them to \textit{reason with and about norms}, agents should be:
\begin{itemize}
    \item able to adopt or drop any number of norm sources as norm instances;
    \item able to qualify observations about their environment as normatively relevant updates, and conversely to respond to normative events by acting accordingly in their environment;
    \item able to query, update, revert and reset a normative state of any norm
instance;
    \item able to receive and process or ignore normative events (e.g. new claims and
liberties)
    \item able to follow or violate normative conclusions (e.g. obligations) or query
responses (e.g. permissions and prohibitions)
    \item able to modify any of the above abilities at run-time.
\end{itemize}

Based on these requirements, the concept of normative advisors is introduced. Normative advisors are entities that enable the social agents to communicate with one or more external norm reasoners. Then, the tasks of maintaining an institutional perspective (state) and reasoning about specific sets of norms is delegated to the advisors while the social agent can maintain its autonomy regardless of what norm sources it has adopted. ASC2's modularity vastly helps in facilitating the implementation of this approach, but it is by no means exclusive to it and because of the simplicity of the method almost any other BDI framework can be utilized in the same manner. Guidelines on how this can be done in other frameworks and what are the requirements, both for the BDI framework and the norm reasoner are also discussed.

\begin{displayquote}
\textbf{Research Question 4:} \textit{How can we model desires and preferences of agent?}
\end{displayquote}

To be effective for the purpose of policy-making requires agent models that to some extent manifest human-like behavior. For \textit{traceability} and \textit{explainability} reasons, decision-making concerning actions need to be analysed alongside the actions themselves (e.g. For which purpose the agent is asking access to the resource? On which basis the infrastructure is granting access?). Furthermore, modelling agents manifest traceable and explainable decision-making whilst requiring them to reason with norms is a challenging task. It is very easy to see that an agent's decisions can be in conflict a set of societal norms, two sets of norms adopted by the agent can be in conflict with each-other, or even agent's desires or goals alone can result in non-trivial conflicts. To address these issues, this dissertation proposes giving the agent designer the ability to encode conditional and context-based higher level decision-making rules in agent programs in form of explicit preferences.

Preferences play a crucial role in decision-making \cite{Pigozzi2016}. Several models of preferences have been presented in the literature (e.g. on decision-making, planning, etc.), with various levels of granularity and expressiveness~\cite{Baier2008,Brafman2009}. On a higher level, preference representation methods can be divided into \textit{quantitative} and \textit{qualitative}~\cite{Baier2008}. The most straightforward \textit{quantitative} approaches are based upon \textit{utility theory} and related forms of decision theory. In \cite{Cranefield2017} one can find some examples of integration of these types of preferences in a BDI architecture. 

Although quantitative approaches have clear computational advantages, they also suffer from the non-trivial issue of elicitation: translating users' preferences into utility functions. This explains the existence of a family of \textit{qualitative} or hybrid solutions, as LPP \cite{Bienvenu2006} and PDDL3 \cite{Gerevini2005}. Some preference models, as CP-nets (qualitative) \cite{Boutilier2004} and GAI networks (quantitative) \cite{Gonzales2004}, have been specifically introduced for taking into account dependencies and conditions between preferences via \textit{compact representations} \cite{Pigozzi2016}, highly relevant in domains with a large number of features.

The strong support in the decision-making literature for \textit{compact representations} of verbalized preferences---as for instance those captured e.g. by CP-nets \cite{Boutilier2004}---motivates their use in computational agents, especially in applications in which agents are deemed to reproduce human behaviour, aiming to capture intentional characterizations of (computational) behaviour of computational agents in data-sharing infrastructures in support of policy-making and regulation activities.  % Agent-based programming, by looking at computational agents as intentional agents, provides this level of abstraction available \textit{by design}.

Chapter~\ref{ch:preferences}, building upon author's previous works~\cite{Mohajeri2019,Mohajeri2020}, introduces an approach for integrating CP-Net preferences into BDI agents at the language level, specifically utilized in abstract goal refinement. This allows the agents to take an abstract goal and concretize it in an incremental manner through a decision-making process based on conditional and context-based preferences explicitly defined by the designer. These conditions and contexts can include observations over the environment, agent's factual and beliefs, adopted norms --the institutional view over the environment---, and external events. 

Introducing explicit preferences in BDI scripts brings three advantages: (1) It increases the \textit{representational depth}, capturing what is the rationale behind the priority in plan selection; (2) It makes agent models more readable and \textit{explainable}, as choices are in principle transparently derived from the preferential structure; (3) It makes the programs more \textit{reusable}: it is plausible  that agents (e.g. representatives of organizations) in a certain domain might share the same procedural knowledge even when having different preferences, as much as that  agents might change their policy without changing their procedural knowledge.





\begin{displayquote}
\textbf{Research Question 5:} \textit{How to make agent-based modelling a usable approach for policy-makers and designers?} 
\end{displayquote}




\section{Limitations}

\textbf{still some computational limitations specially in logic and norm reasoning}

\textbf{High flexibility of the framework means losing some of the logical reasoning power}: When agents can just run any arbitrary piece of code as the polyglot allows, we lose the abilities like having classical model-checking. Note at the PRIMA papers for this and 3APL as example of a framework that goes on the other side of the spectrum.

\textbf{Still very hard to do real-world norm sources, e.g., GDPR. Encoding is a big task}

\section{Future Works} 

Machine learning approaches

Probability-based decision-making

Inclusion of business models and compliance-management frameworks

Complete the model-to-executable process