\chapter{The order of things}
According to ILLC standards your dissertation should meet a limited number
of requirements concerning its organization and layout. You need hardly 
worry about details concerning the layout as these are handled by the
ILLC Dissertation Style file. The following describes how your dissertation
should be organized.

\section{The cover}
The ILLC Dissertation Style only prescribes the font, size and location 
of the title and author on the cover page. Besides this you are free to 
design your own cover.

Dissertations formatted according to ILLC standards have a spine displaying
the authors name, the title of the dissertation, and the ILLC logo.
There is a file called {\tt guide\_spine.tex} to help you format 
your spine text.


\section{The front matter}
The front matter has Roman page numbers (this is achieved by
specifying the command \verb|\pagenumbering{roman}| after the 
\verb|\begin{document}| declaration). The front matter should contain 
the following material in the following order:
\begin{enumerate}
\item[i]
``franse pagina'' containing nothing but the title of your dissertation
\item[ii]
the ``ILLC page'' containing the logo and address of ILLC
\item[iii]
the title page containing the text prescribed by the university
\item[iv]
this page contains the following information in the following order:
	\begin{itemize}
	\item
	name and address of your promotor (es)
	\item
	when appropriate, an acknowledgment to NWO or its subfoundations
	\item
	CIP-gegevens (optional), cataloguing data for the National Library
	\item
	a copyright notice
	\item
	information concerning the production of your dissertation
	\item
	the ISBN code
	\end{itemize}
\item[v] (optional)
dedication
\item[v] (or vii)
table of contents
\item[vii] (or ix)
Acknowledgments, specified by \verb|\acknowledgments|.
\end{enumerate}
The file called  \verb|guide_front.tex| helps you format
the front matter of your dissertation.
Pages (iii) and (iv) will be sent to you by the Bureau Pedel (Office of the Beadle).



\section{The body of your text}
This section contains some information about organizing the main
text of your dissertation.

\paragraph*{Headings.}
Headings will be automatically generated by the following codes
\begin{verbatim}
  \chapter
  \section
  \subsection
  \subsubsection
  \paragraph
\end{verbatim}
The headings produced by \verb|\paragraph| and \verb|\subparagraph| 
need to be punctuated at the end,
as they are followed by the body of the (sub-)paragraph.

\paragraph*{Theorem-like environments.}
In addition to the above headings your text may be structured 
by theorem-like environments, like lemmas, propositions, conjectures, \ldots .
The following theorem-like environments are predefined by the ILLC Disseration 
Style file: \verb|theorem|, \verb|lemma|, \verb|corollary|, \verb|conjecture|, 
\verb|proposition|, \verb|definition|, \verb|remark|, 
\verb|example|, \verb|convention|, \verb|fact| and \verb|question|.
They are defined to be numbered consecutively, i.e. typing
\begin{verbatim}
\begin{lemma}
This is a lemma
\end{lemma}
\begin{proof}
This is a proof\\
With two lines
\end{proof}
\begin{question}
Is this a question?
\end{question}
\end{verbatim}
produces
\begin{lemma}
This is a lemma
\end{lemma}
\begin{proof}
This is a proof\\
With two lines
\end{proof}
\begin{question}
Is this a question?
\end{question}

A number of theorem-like environments have italicized text:
\verb|theorem|, \verb|lemma|, \verb|corollary|, \verb|conjecture|
and \verb|proposition|. All other pre-defined environments have roman text.
Inside theorem-like environments text may be emphasized by
using \verb|\em|. (In environments with italicized text such as lemma
and theorems this will produce text in roman type style; in 
environments with roman text this produces italicized text.)
As a rule of thumb you should always emphasize the terms being
defined in a definition.

\paragraph*{Special signs and characters.}
\newcommand{\AmSTeX}{%
{$\cal A$}\kern-.1667em\lower.5ex\hbox
  {$\cal M$}\kern-.125em{$\cal S$}-\TeX
}
You may need to use special signs. The available ones are listed
in the \LaTeX{} {\em User's Guide \& Reference Manual\/}, pp.~44 ff.
If you need other symbols than those, you could use the symbols
of the \AmSTeX{} fonts. The  \AmSTeX{} fonts also contain gothic letters
and `blackboard bold' characters such as ${\rm I}\hskip -.3pt{\rm N}$. Consult
your local \TeX{} wizzard for instructions on using the \AmSTeX{} fonts.

\paragraph*{Splitting your input}
Rather than putting the whole input of a document in a single file, you
may wish to split it into several smaller ones.
There will always be one file that is the {\em root} file; it is the one
whose name you type when you run \LaTeX{}.
The root file of the document you are reading is called \verb|guide.tex|.
Other files may be `included' by the commands \verb|\input| and \verb|\include|.
The command \verb|\input{filename}| causes \LaTeX{} to insert the contents
of the file \verb|filename.tex| right at the current spot in your manuscript.
The command \verb|\include{filename}| does the same, except that the
included text will begin and end on its own page (i.e. an automatic
\verb|\clearpage| command is issued at the beginning and end of the included
file).
Additionally, this allows the use of the \verb|\includeonly| command
(see the paragraph on saving paper).
The \verb|\include| command is the preferred way to include a file containing,
for instance, the text of a single chapter.

\section{The end matter}
The end matter should at least contain a Bibliography, a Samenvatting,
an Abstract
and a list of previous publications in the ILLC Dissertation Series.
Note that both a dutch summary and an english abstract are obligatory
in english dissertations, according to UvA promotion regulations.
Preferably your dissertation also contains an Index.
In addition it may contain
Appendices, a List of Symbols and your Curriculum Vitae. According to ILLC
standards the material should be included in the following order:
\begin{itemize}
\item
Appendices (optional), see pp.\ 23, 158 of the
  \LaTeX{} {\em User's Guide \& Reference Manual\/} on how to create
  appendices 
\item
Bibliography (obligatory), specified by 
\begin{verbatim}
  \begin{thebibliography}{XX}
    <your list of \bibitems>
  \end{thebibliography}
\end{verbatim}
\item
Index, specified by
\begin{verbatim}
  \begin{theindex}
    <your list of entries>
  \end{theindex}
\end{verbatim}
\item
List of Symbols (optional), specified by
\begin{verbatim}
  \begin{thesymbols}
    <your list of symbols>
  \end{thesymbols}
\end{verbatim}
\item
Samenvatting (obligatory), specified by
\begin{verbatim}
  \samenvatting
    <your Samenvatting>
\end{verbatim}
\item
Abstract (obligatory), specified by
\begin{verbatim}
  \abstract
    <your Abstract>
\end{verbatim}

\item
Curriculum Vitae (optional), specified by
\begin{verbatim}
  \curriculum
    <your CV>
\end{verbatim}
\item
List of previous publications in the ILLC Dissertation Series (obligatory), 
specified by
\begin{verbatim}
  \pagestyle{empty}

\noindent
{\em Titles in the ILLC Dissertation Series:}

\newcommand{\illcpublication}[3]{\item[ILLC #1: ]{\bfseries #2}\\{\em #3}}

\begin{list}{}{ \settowidth{\leftmargin}{ILL}
		\setlength{\rightmargin}{0in}
		\setlength{\labelwidth}{\leftmargin}
		\setlength{\labelsep}{0in}
}

\illcpublication{DS-2016-01}{Ivano A. Ciardelli}{Questions in Logic}
\illcpublication{DS-2016-02}{Zoé Christoff}{Dynamic Logics of Networks: Information Flow and the Spread of Opinion}
\illcpublication{DS-2016-03}{Fleur Leonie Bouwer}{What do we need to hear a beat? The influence of attention, musical abilities, and accents on the perception of metrical rhythm}
\illcpublication{DS-2016-04}{Johannes Marti}{Interpreting Linguistic Behavior with Possible World Models}
\illcpublication{DS-2016-05}{Phong Lê}{Learning Vector Representations for Sentences - The Recursive Deep Learning Approach}
\illcpublication{DS-2016-06}{Gideon Maillette de Buy Wenniger}{Aligning the Foundations of Hierarchical Statistical Machine Translation}
\illcpublication{DS-2016-07}{Andreas van Cranenburgh}{Rich Statistical Parsing and Literary Language}
\illcpublication{DS-2016-08}{Florian Speelman}{Position-based Quantum Cryptography and Catalytic Computation}
\illcpublication{DS-2016-09}{Teresa Piovesan}{Quantum entanglement: insights via graph parameters and conic optimization}
\illcpublication{DS-2016-10}{Paula Henk}{Nonstandard Provability for Peano Arithmetic. A Modal Perspective}
\illcpublication{DS-2017-01}{Paolo Galeazzi}{Play Without Regret}
\illcpublication{DS-2017-02}{Riccardo Pinosio}{The Logic of Kant's Temporal Continuum}
\illcpublication{DS-2017-03}{Matthijs Westera}{Exhaustivity and intonation: a unified theory}
\illcpublication{DS-2017-04}{Giovanni Cinà}{Categories for the working modal logician}
\illcpublication{DS-2017-05}{Shane Noah Steinert-Threlkeld}{Communication and Computation: New Questions About Compositionality}
\illcpublication{DS-2017-06}{Peter Hawke}{The Problem of Epistemic Relevance}
\illcpublication{DS-2017-07}{Aybüke Özgün}{Evidence in Epistemic Logic: A Topological Perspective}
\illcpublication{DS-2017-08}{Raquel Garrido Alhama}{Computational Modelling of Artificial Language Learning: Retention, Recognition \& Recurrence}
\illcpublication{DS-2017-09}{Miloš Stanojević}{Permutation Forests for Modeling Word Order in Machine Translation}
\illcpublication{DS-2018-01}{Berit Janssen}{Retained or Lost in Transmission? Analyzing and Predicting Stability in Dutch Folk Songs}
\illcpublication{DS-2018-02}{Hugo Huurdeman}{Supporting the Complex Dynamics of the Information Seeking Process}
\illcpublication{DS-2018-03}{Corina Koolen}{Reading beyond the female: The relationship between perception of author gender and literary quality}
\illcpublication{DS-2018-04}{Jelle Bruineberg}{Anticipating Affordances: Intentionality in self-organizing brain-body-environment systems}
\illcpublication{DS-2018-05}{Joachim Daiber}{Typologically Robust Statistical Machine Translation: Understanding and Exploiting Differences and Similarities Between Languages in Machine Translation}
\illcpublication{DS-2018-06}{Thomas Brochhagen}{Signaling under Uncertainty}
\illcpublication{DS-2018-07}{Julian Schlöder}{Assertion and Rejection}
\illcpublication{DS-2018-08}{Srinivasan Arunachalam}{Quantum Algorithms and Learning Theory}
\illcpublication{DS-2018-09}{Hugo de Holanda Cunha Nobrega}{Games for functions: Baire classes, Weihrauch degrees, transfinite computations, and ranks}
\illcpublication{DS-2018-10}{Chenwei Shi}{Reason to Believe}
\illcpublication{DS-2018-11}{Malvin Gattinger}{New Directions in Model Checking Dynamic Epistemic Logic}
\illcpublication{DS-2018-12}{Julia Ilin}{Filtration Revisited: Lattices of Stable Non-Classical Logics}
\illcpublication{DS-2018-13}{Jeroen Zuiddam}{Algebraic complexity, asymptotic spectra and entanglement polytopes}
\illcpublication{DS-2019-01}{Carlos Vaquero}{What Makes A Performer Unique? Idiosyncrasies and commonalities in expressive music performance}
\illcpublication{DS-2019-02}{Jort Bergfeld}{Quantum logics for expressing and proving the correctness of quantum programs}
\illcpublication{DS-2019-03}{András Gilyén}{Quantum Singular Value Transformation \& Its Algorithmic Applications}
\illcpublication{DS-2019-04}{Lorenzo Galeotti}{The theory of the generalised real numbers and other topics in logic}
\illcpublication{DS-2019-05}{Nadine Theiler}{Taking a unified perspective: Resolutions and highlighting in the semantics of attitudes and particles}
\illcpublication{DS-2019-06}{Peter T.S. van der Gulik}{Considerations in Evolutionary Biochemistry}
\illcpublication{DS-2019-07}{Frederik Möllerström Lauridsen}{Cuts and Completions: Algebraic aspects of structural proof theory}
\illcpublication{DS-2020-01}{Mostafa Dehghani}{Learning with Imperfect Supervision for Language Understanding}
\illcpublication{DS-2020-02}{Koen Groenland}{Quantum protocols for few-qubit devices}
\illcpublication{DS-2020-03}{Jouke Witteveen}{Parameterized Analysis of Complexity}
\illcpublication{DS-2020-04}{Joran van Apeldoorn}{A Quantum View on Convex Optimization}
\illcpublication{DS-2020-05}{Tom Bannink}{Quantum and stochastic processes}
\illcpublication{DS-2020-06}{Dieuwke Hupkes}{Hierarchy and interpretability in neural models of language processing}
\illcpublication{DS-2020-07}{Ana Lucia Vargas Sandoval}{On the Path to the Truth: Logical \& Computational Aspects of Learning}
\illcpublication{DS-2020-08}{Philip Schulz}{Latent Variable Models for Machine Translation and How to Learn Them}
\illcpublication{DS-2020-09}{Jasmijn Bastings}{A Tale of Two Sequences: Interpretable and Linguistically-Informed Deep Learning for Natural Language Processing}
\illcpublication{DS-2020-10}{Arnold Kochari}{Perceiving and communicating magnitudes: Behavioral and electrophysiological studies}
\illcpublication{DS-2020-11}{Marco Del Tredici}{Linguistic Variation in Online Communities: A Computational Perspective}
\illcpublication{DS-2020-12}{Bastiaan van der Weij}{Experienced listeners: Modeling the influence of long-term musical exposure on rhythm perception}
\illcpublication{DS-2020-13}{Thom van Gessel}{Questions in Context}
\illcpublication{DS-2020-14}{Gianluca Grilletti}{Questions \& Quantification: A study of first order inquisitive logic}
\illcpublication{DS-2020-15}{Tom Schoonen}{Tales of Similarity and Imagination. A modest epistemology of possibility}
\illcpublication{DS-2020-16}{Ilaria Canavotto}{Where Responsibility Takes You: Logics of Agency, Counterfactuals and Norms}
\illcpublication{DS-2020-17}{Francesca Zaffora Blando}{Patterns and Probabilities: A Study in Algorithmic Randomness and Computable Learning}
\illcpublication{DS-2021-01}{Yfke Dulek}{Delegated and Distributed Quantum Computation}
\illcpublication{DS-2021-02}{Elbert J. Booij}{The Things Before Us: On What it Is to Be an Object}
\illcpublication{DS-2021-03}{Seyyed Hadi Hashemi}{Modeling Users Interacting with Smart Devices}
\illcpublication{DS-2021-04}{Sophie Arnoult}{Adjunction in Hierarchical Phrase-Based Translation}
\illcpublication{DS-2021-05}{Cian Guilfoyle Chartier}{A Pragmatic Defense of Logical Pluralism}
\illcpublication{DS-2021-06}{Zoi Terzopoulou}{Collective Decisions with Incomplete Individual Opinions}
\illcpublication{DS-2021-07}{Anthia Solaki}{Logical Models for Bounded Reasoners}
\illcpublication{DS-2021-08}{Michael Sejr Schlichtkrull}{Incorporating Structure into Neural Models for Language Processing}
\illcpublication{DS-2021-09}{Taichi Uemura}{Abstract and Concrete Type Theories}
\illcpublication{DS-2021-10}{Levin Hornischer}{Dynamical Systems via Domains: Toward a Unified Foundation of Symbolic and Non-symbolic Computation}

\end{list}
\end{verbatim}
\end{itemize}
The end matter of this document has been split into separate files,
\verb|included| in the main file.
In this document, each file except for {\tt illcdissertations.tex} 
contains a copy of the corresponding entry from the overview above.

\section{The spine}
You can use the file {\tt guide\_spine.tex} to
typeset the text on the spine of your dissertation. This
text should consist of your name, the title of your dissertation,
and the ILLC logo.

The file {\tt guide\_spine.tex} produces the text for the spine
of your dissertation in a number of sizes. Let your competent printer
choose the most appropriate size.





