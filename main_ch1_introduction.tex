\chapter{Introduction}


As (software) systems are getting more integrated with our day-to-day and have more impact of the society, there is a requirement for these systems to be designed in a way that are more aligned with norms of society (or policies, or regulations) meaning norms and policies should be considered as part of system development cycle by the designers. On the other hand, systems affect the society meaning they are also affecting the norms, so they should be already considered when policy-makers are analyzing existing norms or implementing new ones. Taking modelling and model execution as a principal approach to designing both systems and norms, intuitively, requires the designers to be able to model these concepts, which include executable models of norms, models of the system, and, the social setting and its actors. The expressivity and flexibility required to encompass these aspects is not trivial. It is no longer just the models of a system, a social setting, and set of norms, but, it is a system that is utilized by social actors and governed by norms, a social setting that utilize and monitor a software system and decide upon, modify, analyze and enforce the norms, and, norms that regulate the system and the social setting that have impacts on both. 


The purpose of this thesis is to closely study the interactions between these concepts, search for fundamental gaps in current methodologies utilized by system designers and policy makers, with the goal of finding approaches for creating more expressive and tangible models that are also flexible enough for specifying different scenarios and use-cases in different fields




\section{Motivation and Research Questions}

\section{Approach and Scope of this Thesis}

\section{Policy-Making Scenarios}

\section{Research Collaborations}
Peter, Xin, Lu-Chi, Giovanni, Lu, Sarah, and, Thomas go here.